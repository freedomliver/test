\documentclass[%
 reprint,
%superscriptaddress,
%groupedaddress,
%unsortedaddress,
%runinaddress,
%frontmatterverbose, 
%preprint,
%showpacs,preprintnumbers,
%nofootinbib,
%nobibnotes,
%bibnotes,
 amsmath,amssymb,
 aps,
%pra,
%prb,
%rmp,
%prstab,
%prstper,
%floatfix,
]{revtex4-1}

\usepackage{graphicx}% Include figure files
\usepackage[UTF8]{ctex}
\usepackage{dcolumn}% Align table columns on decimal point
\usepackage{bm}% bold math

\begin{document}


\preprint{APS/123-QED}

\title{天体物理概论笔记}
\author{jincheng}
 \email{jincheng@sjtu.edu.cn}
\affiliation{%
 SJTU university \textbackslash\textbackslash
}%

\date{\today}

\begin{abstract}
对天体物理概论学习的笔记总结。
\end{abstract}

\pacs{Valid PACS appear here}
\maketitle

\section{\label{sec:level1}基本天体物理量及测量}
	
\paragraph{基本天体物理量}大小(黑体辐射半径)、质量(质光关系和位力定理)、温度(色温度)、辐射功率(光度)、化学成分()光谱型)以及年龄()星团郝罗图对比)和距离(视差和图形对比)等特征量。

\paragraph{恒星}恒星是指能够通过核反应而发光的天体;是形成各种宇宙结构的基本组元。

\paragraph{视星等}地球上或空间探测器观测到的天体亮度的一种量度;仅仅表示观测到的恒星亮度。亮度即单位面积接收到的辐射流量。

\subsubsection{视星等和亮度关系}
$$\frac{E_1}{E_2} = 100^{\frac{m_2 - m_1}{5}}$$
其中$m_1,m_2$为两颗星的视星等,$E_1,E_2$为两颗星的亮度。变换一下形式得,$$m_2 - m_1 = - 2.5lg \frac{E_2}{E_1}$$

\paragraph{}不同的测量方法或观测波段对同一天体可以给出不同的星等,这就产生了不同的星等系统和光度系统。人眼所观测到的星等称为目视星等$m_v$,照相底片所测到的星等称为照相星等$m_p$,用黄绿色滤光片配合照相底片得到和人眼大致相同的仿视星等$m_{pv}$,安装在望远镜终端的光电光度计测得的光电星等,以及辐射星等、热星等;等等。

\paragraph{绝对星等M} 把视星等为m的恒星移到10秒差距的距离时的视星等称为绝对星等。 秒差距(1$pc \approx 3.26 ly \approx 3.1\times10^{18} cm$)

\subsubsection{星等和距离的关系}
观测到的亮度与距离平方成反比,所以有$\frac{E_{10}}{E} = \frac{r^2}{10^2}$,由第一个公式得$m - M = - 2.5lg \frac{E}{E_{10}} = 5lgr - 5$;所以$$M = m+5 - 5lgr$$

\paragraph{大气消光}星光被地球大气吸收和散射的结果。

\paragraph{光度}恒星整个表面发射的所有波段的总辐射功率。光度是恒星本身所固有的、表征其辐射本领的量。太阳的光度为:$L_{\bigodot} = 3.8\times 10^{33} erg/s$

\subsubsection{绝对星等M和光度L的关系}
$$lg\frac{L}{L_{\bigodot}} = \frac{1}{2.5}(M_{\bigodot} - M)$$
光度L是指恒星的辐射总量,亮度E是指测量时单位面积收到的辐射量。二者基本满足同一关系。

\subsubsection{维恩位移定律}
$$\lambda_{max} = \frac{b}{T}$$
其中波长$\lambda_{max}$对应黑体辐射能量分布曲线最大值处,$T$为黑体温度。$b$是维恩位移常量($b=2.898\times 10^{-3} m \cdot K$)。

\paragraph{色指数C}同一颗星在不同波段的测光星等之差成为色指数。常用照相星等和目视星等(或仿视星等)来定义色指数。

\subsubsection{色温度和色指数近似关系}
$$T = \frac{7200}{C + 0.64}K$$
用色指数得到的T称为色温度。使用不同的滤片会得到不同测光系统;例如,当用B(蓝)-V(可见光中心)表示时有
$$T = \frac{7090}{B - V + 0.71}K$$

\paragraph{有效温度}把与恒星光度L相同的绝对黑体温度称为有效温度$T_e$(表示辐射的有效温度)。($L = 4\pi R^2 \sigma T_e^4$)得到有效温度为$$T_e = \big( \frac{L}{4 \pi R^2 \sigma} \big)^{1/4} $$
其中,$R$为恒星半径,$\sigma$为斯特潘-玻尔兹曼常量。

\paragraph{星际红化与星际消光}空间中的大量气体和尘埃对短波散射强烈,使星光偏红;气体和尘埃还会吸收或屏蔽光线,使星光变暗。

\subsubsection{基尔霍夫定律}
(1)每一种化学元素在高温下都能产生辐射而发出独特的明线光谱
(2)在低温下每一种元素都可以吸收自己能够发出的谱线,在光谱对应位置形成暗线。

\paragraph{}哈弗分类法是恒星光谱基于温度(颜色)的一元分类法。主序列为$O--B--A--F--G--K--M$。此外还有MK分类系统:在哈弗分类法后面加上罗马数字表示光度(I.超巨星;II.亮巨星;III.正常巨星;IV.亚巨星;V.主序星(矮星);VI.亚矮星;VII.白矮星。)也就是温度+辐射强度。

\subsubsection{玻尔兹曼能级分布}
$$\frac{n_j}{n_i} = \frac{g_j}{g_i}exp\big(- \frac{E_j - E_i}{kT}\big) $$
其中,$n_j$和$n_i$分别表示能级i和j的原子数,$E_i$和$E_j$分别表示对应能级的能量;T为气体温度,$g_i$和$g_j$分别为相应能级的的统计权重。(反应简并度)k为玻尔兹曼常数,T为气体温度,kT = 0.025852 eV(T = 300K。
以氢原子为例,第一激发态和原子基态能量之差为10.2eV,简并度之比为4,在太阳表面温度(5700K)时,第一激发态与基态数目比约为$4\times 10^{}-9$,当温度升高,更多氢原子将处于激发态。

\subsubsection{萨哈方程}
\paragraph{电离原子表示法}I表示中性原子,II为一次电离,III为二次电离。或者用加号表示:HI=H,$HI=H^+$,$Ca\uppercase\expandafter{\romannumeral3} = Ca^{++}$。

当温度足够高,原子之间的碰撞将会使原子电离,电离原子复合会发射电子,达到平衡时,
$$\frac{n_e n(X_{r+1})}{n(X_r)} = \frac{2g_{r+1}}{g_r} \frac{(2\pi m_e k T)^3/2}{h^3} exp\big(- \frac{E_r}{kT}\big) $$
其中,$n_e$为自由电子数密度,$n(X_{r+1})$和$n(X_r)$分别表示元素X的r+1次和r次电离态的数密度,$g_{r+1}$和$g_r$分别表示$X_{r+1}$和$X_r$的原子基态简并度,2表示电子自旋简并度$g_e$,$E_r$表示从$X_r$到$X_{r+1}$的电离能。可以看出除非气体全部由氢原子组成,哈萨方程非常复杂。

元素原子的电离能越大,激发态与基态的能级之差也就越大。因此,具有较高电离能的元素谱线一定出现在较高温度的光谱型中,谱线强度随着温度升高而增强,(激发态原子数目增多)但当温度超过一定值时,元素的电离使得中性原子变少,此时尽管激发态原子在中性原子中占比很大,但是其绝对数量却随着温度增加而减少,总的效果就是谱线强度在达到一个极大值后迅速下降。

\paragraph{郝罗图}横轴表示光谱型(温度、色指数等),纵轴表示光度(或绝对星等)。

从左上到右下的一个狭窄的带称为主星序;它们温度相同时,光度也基本相同。位于郝罗图右上方的是红巨星和红超巨星;它们都是恒星演化到离开主星序后形成的。位于郝罗图左下方的是白矮星。

\subsubsection{恒星等半径线}以黑体辐射的光度和温度、半径之间关系($L = 4\pi R^2 \sigma T_e^4$)以太阳为参照得,$lg\frac{L}{L_{\bigodot}} = 2lg\frac{R}{R_{\bigodot}} + 4lg\frac{T_e}{T_{e\bigodot}}$,根据第三个公式消掉L,再使用太阳的一些常量可以得到$$lg\frac{R}{R_{\bigodot}} = 8.49 - 0.2M - 2lgT_e$$,在$M-lgT_e$图像里是一组平行线。

\paragraph{变星}光度、光谱特征、磁场等物理特性随时间做周期性、半规则或者无规则变化的恒星称为变星。简而言之,亮度有变化的恒星称为变星。

由于双星相互掩食使得亮度周期性变化的称为几何变星或食变星。

由于轨道运动的多普勒效应使得光谱线周期性变化的称为分光变星。

由于自身的物理原因,在较长时间里光度有明显变化的恒星称为物理变星。又分为脉动变星(星体大气层周期性的膨胀和收缩)(包括造父变星、半规则变星、不规则变星和长周期变星等)和爆发变星(超新星、激变变星、早期演化阶段的变星和有延伸气壳的早型变星等等。

对造父变星有周光关系,光变周期越长,光度就越大。$$<Mv> = -2.81lgP - 1.43$$这是目前比较精确的周光关系式,其中$<Mv
$一个周期的平均绝对视星等,P为天文单位的光变周期

\paragraph{新星爆发与超新星爆发}新星爆发实际上一颗白矮星和一颗巨星组成的致密近星在恒星演化的晚期,白矮星把巨星的外层物质吸过去,在表面堆积并最终引爆氢的核聚变,突然爆发并把大量物质抛射出去,使得核聚变范围迅速扩大,亮度迅速增加;新星爆发基本不会破坏双星基本结构。而超新星爆发是走到生命尽头的红巨星自身结构的瓦解,中心物质会坍缩成白矮星、中子星或者黑洞。

天体距离的测定是我们认识宇宙的基础。对于太阳附近的恒星可以用三角视差法进行测量。对于三角视差测不出来的恒星距离,只能想办法使用某些特性作为真实亮度的标准烛光间接测量;对最遥远的天体,只有依靠红移和距离之间的哈勃关系了。

\subsubsection{天体距离的测定}
\paragraph{三角视差法}周年视差简称为视差,因此,距离有,$r = \frac{a}{sin\pi} \approx \frac{a}{\pi} = 206265\frac{a}{\pi''} $其中1弧度等于57.3*60*60角秒,所以1秒差距($pc \approx 3.26 ly \approx 3.1\times10^{18} cm$

分光视差法;是指特殊谱线的强度只随绝对星等而改变。

威尔逊-巴普法;是指特殊光谱线的宽度对数与绝对星等成比例。

主序星重叠法;将未知距离的星团的郝罗图与已知进行比较得到绝对星等。(通过测量几百几千颗恒星的集合——星团的距离来确定恒星大致距离)

造父视差法;用几类造父变星作为标准烛光。
还可以通过将未知天体的光度函数与已知比较,得到距离。

哈勃关系;对河外星系等遥远天体,当谱线红移量z小于1时,其谱线红移z与距离r成正比:$$z = \frac{ \delta \lambda}{\lambda} = \frac{\lambda_0 - \lambda}{\lambda_0} = \frac{H_0}{c}r$$其中,$\lambda_0$和$\lambda$分别表示原来的波长和测得的波长,c为光速,$H_0$哈勃常数($H_0 = 100 h km s^{-1} Mpc^{-1}$),目前的观测值h为0.72;当z大于1时,哈勃关系变得更复杂。

\paragraph{质光关系}对于主序星,光度对的质量有着基本类似的质光关系。可以近似的表示为$$L/L_0 \propto(M/M_{\bigodot})^{\alpha} $$

\begin{array}{ccc}
	$\alpha = 1.8$ & for$M \le 0.3M_{\bigodot}$ &small mass\\$\alpha = 4.0$ & for$ 0.3 M \le M \le 3M_{\bigodot}$ &middle mass\\$\alpha = 2.8$ & for$M \ge 3M_{\bigodot}$ &big mass
\end{array}

\paragraph{位力定理}$2<T> + <V> = 0$,假设系统近似球形,可以写成$M<v^2> - \frac{3}{5}\farc{GM^2}{R} = 0$,进而求得质量。

\paragraph{恒星年龄}恒星年龄和恒星演化模型密切关联。主要参考对比郝罗图上主星序的转向点。

\section{\label{sec:level2}恒星的形成与演化}

\subsubsection{恒星的形成}
\paragraph{}

\end{document}
%
% ****** End of file apssamp.tex ******