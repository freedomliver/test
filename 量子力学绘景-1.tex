% ****** Start of file apssamp.tex ******
%
%   This file is part of the APS files in the REVTeX 4.1 distribution.
%   Version 4.1r of REVTeX, August 2010
%
%   Copyright (c) 2009, 2010 The American Physical Society.
%
%   See the REVTeX 4 README file for restrictions and more information.
%
% TeX'ing this file requires that you have AMS-LaTeX 2.0 installed
% as well as the rest of the prerequisites for REVTeX 4.1
%
% See the REVTeX 4 README file
% It also requires running BibTeX. The commands are as follows:
%
%  1)  latex apssamp.tex
%  2)  bibtex apssamp
%  3)  latex apssamp.tex
%  4)  latex apssamp.tex
%
\documentclass[%
 reprint,
%superscriptaddress,
%groupedaddress,
%unsortedaddress,
%runinaddress,
%frontmatterverbose, 
%preprint,
%showpacs,preprintnumbers,
%nofootinbib,
%nobibnotes,
%bibnotes,
 amsmath,amssymb,
 aps,
%pra,
%prb,
%rmp,
%prstab,
%prstper,
%floatfix,
]{revtex4-1}

\usepackage{graphicx}% Include figure files
\usepackage[UTF8]{ctex}
\usepackage{dcolumn}% Align table columns on decimal point
\usepackage{bm}% bold math
%\usepackage{hyperref}% add hypertext capabilities
%\usepackage[mathlines]{lineno}% Enable numbering of text and display math
%\linenumbers\relax % Commence numbering lines

%\usepackage[showframe,%Uncomment any one of the following lines to test 
%%scale=0.7, marginratio={1:1, 2:3}, ignoreall,% default settings
%%text={7in,10in},centering,
%%margin=1.5in,
%%total={6.5in,8.75in}, top=1.2in, left=0.9in, includefoot,
%%height=10in,a5paper,hmargin={3cm,0.8in},
%]{geometry}

\begin{document}


\preprint{APS/123-QED}

\title{量子力学绘景}% Force line breaks with \\

\author{jincheng}
 \email{jincheng@sjtu.edu.cn}
\affiliation{%
 SJTU university \textbackslash\textbackslash
}%



\date{\today}% It is always \today, today,
             %  but any date may be explicitly specified

\begin{abstract}
量子力学绘景主要包括薛定谔绘景、海森堡绘景和相互作用绘景;其中,对于薛定谔绘景,可观测量算符不随时间演变;态矢量随时间演化。海森堡绘景则是态矢量不随时间演化,算符随时间演化。但是对于大多数情况下,系统都得不到准确解,所以在相互作用绘景下,可以求得系统的微扰解。
\end{abstract}

\pacs{Valid PACS appear here}% PACS, the Physics and Astronomy
                             % Classification Scheme.
%\keywords{Suggested keywords}%Use showkeys class option if keyword
                              %display desired
\maketitle

%%\tableofcontents

\section{\label{sec:level1}薛定谔绘景}
	
在薛定谔绘景里,量子系统的态矢量随着时间流易而演化,而像位置、自旋一类的对应于可观察量的算符则与时间无关。
在薛定谔绘景里,负责时间演化的算符是一种幺正算符,称为时间演化算符。假设时间从${\displaystyle t_{0}}$流易到${\displaystyle t}$,而经过这段时间间隔,态矢量${\displaystyle |\psi (t_0)\rangle }$ 演化为态矢量${\displaystyle |\psi (t)\rangle }$,这时间演化过程以方程表示为$|\psi (t)\rangle =U(t,t_{0})|\psi (t_{0})\rangle$;
其中,$U(t,t_{0})$是时间演化算符。假设系统的哈密顿量H不含时,则时间演化算符为$U(t,t_{0})=e^{-iH(t-t_{0})/\hbar}$


所以,在薛定谔绘景下,态矢量以薛定谔方程演化,(S表示薛定谔绘景)$$i\hbar \frac{\partial}{\partial t} |\psi(t)\rangle_S = H |\psi(t)\rangle_S$$
态矢量表示为:$$|\psi(t)\rangle_S=U(t,t_0)|\psi(t_0)\rangle_S$$所以有,$i\hbar \frac{\partial}{\partial t} U(t,t_0) = H U(t,t_0)$

可观测量表示为:$$\langle Q \rangle_S = \langle \psi(t) | Q | \psi(t) \rangle = \langle\psi(t_0) |U^*Q U| \psi(t_0)\rangle $$

\subsubsection{可观测量和算符}
\paragraph{算符}
算符又称算子,作用于物理系统的状态空间,使得物理系统从某种状态变换为另外一种状态;在量子力学里的算符称为“量子算符”,作用的对象是量子态。量子算符将某量子态映射为另一种量子态。

\paragraph{可观测量}
物理实验中可以观测到的物理量称为可观测量。每一个可观测量,都有其对应的算符(而且是厄密算符)。可观察量的算符也许会有很多本征值与本征态。

\subsubsection{简单的哈密顿量守恒情况}
假设系统的哈密顿量$\mathcal{H}$不含时,则时间演化算符为
$U(t,t_{0})= e^{-iH(t-t_{0})/\hbar}$只能通过泰勒展开级数计算;态矢量为:
$$|\psi(t)\rangle_S=e^{-iH(t-t_{0})/\hbar} |\psi(t_0)\rangle_S$$



\section{\label{sec:level1}海森堡绘景}

在海森堡绘景里,量子系统的态矢量不随着时间流易而演化,而像位置、自旋一类的对应于可观察量的算符则随着时间流易而演化。

在海森堡绘景里,态矢量$|\psi \rangle _{\mathcal {H}}$不含时,而可观察量的算符$A_{\mathcal {H}}$含时,并且满足“海森堡运动方程”:
$$ \frac {\partial }{\partial {t}}A_{\mathcal {H}}={1 \over i\hbar }[A_{\mathcal {H}},H]$$
态矢量表示为:$$|\psi(t) \rangle_H = |\psi(t_0) \rangle_H = |\psi(t_0) \rangle_S$$
可观测量表示为:$$\langle A \rangle_H = \langle \psi(t_0) | A_H(t) | \psi(t_0) \rangle = \langle \psi(t_0) |U^* A_S U |\psi(t_0) \rangle$$

\subsubsection{海森堡运动方程}
从某种角度来看,海森堡绘景比薛定谔绘景显得更为自然,更具有基础性,因为,经典力学分析物体运动所使用的物理量是可观察量,例如,位置、动量等等,而海森堡绘景专注的就是这些可观察量随着时间流易的演化。进一步来看,海森堡绘景表述的量子力学与经典力学的相似可以很容易的观察到:只要将对易算符改为泊松括号,海森堡方程立刻就变成了哈密顿力学里的运动方程,其形式表示为:

$\frac {\partial }{\partial {t}} A = [A,\,H]_{Poisson}$ 
在这里需要通过狄拉克量子化条件$[\ ,\,\ ]_{Poisson}\ \to \ \frac {[\ ,\,\ ]}{i\hbar }$,就可以从哈密顿力学的运动方程得到海森堡运动方程。

\paragraph{通过薛定谔绘景推导海森堡运动方程}
$\begin{aligned}{d \over dt}A_{\mathcal {H}}(t)&={\frac {\partial U^{\dagger }(t,0)}{\partial t}}A_{\mathcal {S}}U(t,0)+U^{\dagger }(t,0)A_{\mathcal {S}}{\frac {\partial U(t,0)}{\partial t}}\\&=-{1 \over i\hbar }U^{\dagger }HA_{\mathcal {S}}U+{1 \over i\hbar }U^{\dagger }A_{\mathcal {S}}HU\\&=-{1 \over i\hbar }U^{\dagger }HUU^{\dagger }A_{\mathcal {S}}U+{1 \over i\hbar }U^{\dagger }A_{\mathcal {S}}UU^{\dagger }HU\\&={1 \over i\hbar }[U^{\dagger }A_{\mathcal {S}}U,U^{\dagger }HU]\\\end{aligned}$

\subsubsection{贝克-豪斯多夫引理}
引理:${e^{B}Ae^{-B}}=A+[B,A]+{\frac {1}{2!}}[B,[B,A]]+{\frac {1}{3!}}[B,[B,[B,A]]]+\cdots$ 。相对应的,对于算符有:$ A_{\mathcal {H}}(t)=A_{\mathcal {H}}(0)+{\frac {it}{\hbar }}[H,A_{\mathcal {H}}(0)]-{\frac {t^{2}}{2!\hbar ^{2}}}[H,[H,A_{\mathcal {H}}(0)]]-{\frac {it^{3}}{3!\hbar ^{3}}}[H,[H,[H,A_{\mathcal {H}}(0)]]]+\cdots$基于此,可以对相关问题进行计算。

\section{\label{sec:level1}相互作用绘景}

相互作用绘景也叫狄拉克绘景,在相互作用绘景里,量子系统的态矢量和可观察量的算符都随着时间流易而演化。
对于实际中遇到的问题,大多数情况下,总是有相互作用,哈密顿量$mathcal_{H}$随时间改变,薛定谔方程都得不到准确解,但是可以将哈密顿量拆分成准确部分和微小摄动,求其微扰解。$H = H_{0,S} + H_{I,S}$

在相互作用绘景里,态矢量表示为:$$|\psi(t) \rangle_I = e^ {iH_{0,S}t\over \hbar}|\psi(t_0) \rangle_S$$
在狄拉克绘景里,态矢量$|\psi (t)\rangle _{\mathcal {I}}\,\!$定义为
$|\psi (t)\rangle _{\mathcal {I}}{\stackrel {def}{=}}e^{iH_{0,\,{\mathcal {S}}}\,t/\hbar }|\psi (t)\rangle _{\mathcal {S}}\,\!$
其中,$|\psi (t)\rangle _{\mathcal {S}}\,\!$是在薛定谔绘景里的态矢量。

由于在薛定谔绘景里, 态矢量$|\psi (t)\rangle _{\mathcal {S}}\,\!$与时间的关系为
$ |\psi (t)\rangle _{\mathcal {S}}=e^{-iH_{\mathcal {S}}\,t/\hbar }|\psi (0)\rangle _{\mathcal {S}}\,\!$所以,在$ H_{0,{\mathcal {S}}},H_{\mathcal {S}}$对易的条件下,可以有
$|\psi (t)\rangle _{\mathcal {I}}=e^{-iH_{1,\,{\mathcal {S}}}\,t/\hbar }|\psi (0)\rangle _{\mathcal {S}}\,\!$。
算符表示为$ A_{\mathcal {I}}(t)\,\!$。定义为$$ A_{\mathcal {I}}(t)=e^{iH_{0,\,{\mathcal {S}}}\,t/\hbar }A_{\mathcal {S}}(t)\,e^{-iH_{0,\,{\mathcal {S}}}\,t/\hbar }\,\!$$
其中,$ A_{\mathcal {S}}(t)\,\!$是在薛定谔绘景里对应的算符。

量子态随时间演化为:$$ i\hbar {\frac {d}{dt}}|\psi (t)\rangle _{\mathcal {I}}=H_{1,\,{\mathcal {I}}}|\psi (t)\rangle _{\mathcal {I}}\,\!$$
算符随时间演化为:$$ i\hbar {\frac {d}{dt}}A_{\mathcal {H}}(t)=\left[A_{\mathcal {H}}(t),\,H\right]\,\!$$



\end{document}
%
% ****** End of file apssamp.tex ******
